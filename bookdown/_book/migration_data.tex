% Options for packages loaded elsewhere
\PassOptionsToPackage{unicode}{hyperref}
\PassOptionsToPackage{hyphens}{url}
%
\documentclass[
]{book}
\usepackage{amsmath,amssymb}
\usepackage{lmodern}
\usepackage{ifxetex,ifluatex}
\ifnum 0\ifxetex 1\fi\ifluatex 1\fi=0 % if pdftex
  \usepackage[T1]{fontenc}
  \usepackage[utf8]{inputenc}
  \usepackage{textcomp} % provide euro and other symbols
\else % if luatex or xetex
  \usepackage{unicode-math}
  \defaultfontfeatures{Scale=MatchLowercase}
  \defaultfontfeatures[\rmfamily]{Ligatures=TeX,Scale=1}
\fi
% Use upquote if available, for straight quotes in verbatim environments
\IfFileExists{upquote.sty}{\usepackage{upquote}}{}
\IfFileExists{microtype.sty}{% use microtype if available
  \usepackage[]{microtype}
  \UseMicrotypeSet[protrusion]{basicmath} % disable protrusion for tt fonts
}{}
\makeatletter
\@ifundefined{KOMAClassName}{% if non-KOMA class
  \IfFileExists{parskip.sty}{%
    \usepackage{parskip}
  }{% else
    \setlength{\parindent}{0pt}
    \setlength{\parskip}{6pt plus 2pt minus 1pt}}
}{% if KOMA class
  \KOMAoptions{parskip=half}}
\makeatother
\usepackage{xcolor}
\IfFileExists{xurl.sty}{\usepackage{xurl}}{} % add URL line breaks if available
\IfFileExists{bookmark.sty}{\usepackage{bookmark}}{\usepackage{hyperref}}
\hypersetup{
  pdftitle={Proyecto Migración},
  pdfauthor={Equipo Investigación},
  hidelinks,
  pdfcreator={LaTeX via pandoc}}
\urlstyle{same} % disable monospaced font for URLs
\usepackage{longtable,booktabs,array}
\usepackage{calc} % for calculating minipage widths
% Correct order of tables after \paragraph or \subparagraph
\usepackage{etoolbox}
\makeatletter
\patchcmd\longtable{\par}{\if@noskipsec\mbox{}\fi\par}{}{}
\makeatother
% Allow footnotes in longtable head/foot
\IfFileExists{footnotehyper.sty}{\usepackage{footnotehyper}}{\usepackage{footnote}}
\makesavenoteenv{longtable}
\usepackage{graphicx}
\makeatletter
\def\maxwidth{\ifdim\Gin@nat@width>\linewidth\linewidth\else\Gin@nat@width\fi}
\def\maxheight{\ifdim\Gin@nat@height>\textheight\textheight\else\Gin@nat@height\fi}
\makeatother
% Scale images if necessary, so that they will not overflow the page
% margins by default, and it is still possible to overwrite the defaults
% using explicit options in \includegraphics[width, height, ...]{}
\setkeys{Gin}{width=\maxwidth,height=\maxheight,keepaspectratio}
% Set default figure placement to htbp
\makeatletter
\def\fps@figure{htbp}
\makeatother
\setlength{\emergencystretch}{3em} % prevent overfull lines
\providecommand{\tightlist}{%
  \setlength{\itemsep}{0pt}\setlength{\parskip}{0pt}}
\setcounter{secnumdepth}{5}
\usepackage{booktabs}
\ifluatex
  \usepackage{selnolig}  % disable illegal ligatures
\fi
\usepackage[]{natbib}
\bibliographystyle{apalike}

\title{Proyecto Migración}
\author{Equipo Investigación}
\date{2022-03-11}

\begin{document}
\maketitle

{
\setcounter{tocdepth}{1}
\tableofcontents
}
\hypertarget{fuentes-de-datos-en-el-triangulo-del-norte-de-centroamuxe9rica-sobre-migraciuxf3n-ocupaciuxf3n-y-empleo}{%
\chapter{Fuentes de datos en el triangulo del norte de Centroamérica sobre migración, ocupación y empleo}\label{fuentes-de-datos-en-el-triangulo-del-norte-de-centroamuxe9rica-sobre-migraciuxf3n-ocupaciuxf3n-y-empleo}}

\hypertarget{el-salvador}{%
\section{El Salvador}\label{el-salvador}}

\hypertarget{transparencia}{%
\subsection{Transparencia}\label{transparencia}}

El Salvador cuenta con un instituto especializado en migración que publica datos sobre este tema: la Dirección General de Migración y Extranjería de El Salvador (DGME). El apartado de estadísticas del sitio oficial no tiene ningún contenido. Al parecer, los datos se encuentran en el sitio de Transparencia del gobierno. Se recomienda hacer uso de los filtros o el buscador para optimizar las búsquedas.

\url{https://www.transparencia.gob.sv/institutions/dgme/documents/estadisticas}
\url{https://www.migracion.gob.sv/estadisticas/}
\url{https://www.transparencia.gob.sv/institutions/dgme/documents/estadisticas}

Contiene información sobre salvadoreños retornados por país, salidas migratorias por diferentes puestos de manera anual.

\hypertarget{direcciuxf3n-general-de-estaduxedstica-y-censos-de-el-salvador}{%
\subsection{Dirección General de Estadística y Censos de El Salvador}\label{direcciuxf3n-general-de-estaduxedstica-y-censos-de-el-salvador}}

La DIGESTYC es una institución estatal de El Salvador encargada de la elaboración de estudios estadístico. Esta oficina pública depende del Ministerio de Economía. El sitio de la Dirección General de Estadísticas y Censos de El Salvador tiene información sobre estadísticas económicas, sociales, cartografía social y actualización de datos sobre el país.

En el sitio oficial de DIGESTYC podemos encontrar distintos apartados: Institución, servicio, temas, novedades, etc. En el apartado de ``Temas'' se divide en subtemas que incluyen: estadísticas económicas, sociales, cartografía censal, etc. Dentro de las Estadísticas Sociales, se encuentran el Censo de Población y Vivienda y proyecciones.

\url{http://www.digestyc.gob.sv/index.php/temas/des/poblacion-y-estadisticas-demograficas.html}
\url{https://www.transparencia.gob.sv/institutions/minec/documents/estadisticas}
\url{http://www.digestyc.gob.sv/index.php/temas/des/poblacion-y-estadisticas-demograficas/censo-de-poblacion-y-vivienda.html}
\url{http://www.digestyc.gob.sv/index.php/temas/des/poblacion-y-estadisticas-demograficas/censo-de-poblacion-y-vivienda/publicaciones-censos.html}
\url{http://www.digestyc.gob.sv/index.php/temas/des/poblacion-y-estadisticas-demograficas/censo-de-poblacion-y-vivienda/poblacion-censos.html}

En los links anteriores se encuentra todo lo relacionado a:

\begin{itemize}
\item
  Censos: Población, edificios y vivienda, Agropecuario e Industrial y Comercial
\item
  Estadísticas continuas: Geográficas y Meteorológicas, Demográficas y de Salud Pública, Educacionales y Culturales, de Trabajo y Previsión social, Judiciales y Políticas, de Construcciones, Transporte y Vías de Comunicación, Agropecuarias, Industriales, de Comercio Exterior e Interior, Financieras, Fiscales y Administrativas, de Ingreso Nacional y Costo de Vida
\end{itemize}

\hypertarget{ehpm-del-digestyc}{%
\subsection{EHPM del DIGESTYC}\label{ehpm-del-digestyc}}

\begin{itemize}
\tightlist
\item
  Situación socioeconómica de los Hogares Salvadoreño
\item
  Cada 5 años
\end{itemize}

\hypertarget{censo-del-digestyc}{%
\subsection{Censo del DIGESTYC}\label{censo-del-digestyc}}

\begin{itemize}
\item
  Características de la Vivienda
\item
  Características del Hogar
\item
  Características de las Personas que Conforman el Hogar
\item
  Mortalidad y Emigración
\item
  Datos de las Personas del Hogar.
\end{itemize}

\hypertarget{observatorio-de-estaduxedstica-de-guxe9nero-del-digestyc}{%
\subsection{Observatorio de estadística de género del DIGESTYC}\label{observatorio-de-estaduxedstica-de-guxe9nero-del-digestyc}}

\begin{itemize}
\item
  Información sobre uso de tiempo
\item
  Género y migración
\end{itemize}

\url{http://aplicaciones.digestyc.gob.sv/observatorio.genero/migraciones/index.aspx}

\hypertarget{otros}{%
\subsection{Otros}\label{otros}}

Además, existen distintos programas, principalmente internacionales o regionales, que publican una gran cantidad de los datos sobre migración. La mayoría de los programas son gestionados por organizaciones internacionales o instituciones nacionales, particularmente de estadística. La mayoría son encuestas sobre Demografía u Hogares. En El Salvador existen los siguientes:

DTM Matriz de Seguimiento de Desplazamiento. Es un sistema informativo de la Organización Internacional. Esta matriz es una herramienta de la Organización Internacional para las Migraciones que brinda información estadística que didáctica para mejorar la comprensión sobre características y necesidades de grupos inmersos en dinámicas de movilidad. Herramienta clave para la recopilación de datos primarios y la elaboración de informes sobre movilidad humana, tanto a escala nacional regional y global.

\url{https://dtm.iom.int/el-salvador}

\url{https://mic.iom.int/webntmi/descargas/sv/2017/OIM-DTMsv.pdf}
\url{https://displacement.iom.int/sites/default/files/public/DOE\%20Infosheet\%20-\%20201911_SP.pdf}

\url{https://reliefweb.int/report/el-salvador/dtm-el-salvador-flow-monitoring-survey-profiles-and-humanitarian-needs-migrants}

Encuesta de Hogares de Propósitos Múltiples: Realizada por el Ministerio de Economía, a través de la Dirección General de Estadística y Censos (DIGESTYC).
La EHPM es un instrumento estadístico con el que cuenta el país, para proporcionar información sobre la situación socioeconómica de los Hogares salvadoreños.

\url{http://www.digestyc.gob.sv/index.php/temas/des/ehpm.html}

\url{http://www.digestyc.gob.sv/index.php/novedades/avisos/965-ya-se-encuentra-disponible-la-encuesta-de-hogares-de-propositos-multiples-2019.html}

Iniciativa de Gestión de Información de Movilidad Humana en el Triángulo Norte. La OIM puso en marcha un proyecto financiado por la USAID denominado Iniciativa de Gestión de Información de Movilidad Humana en el Triángulo Norte (NTMI por sus siglas en inglés), cuyo objetivo es fortalecer las capacidades de los gobiernos de El Salvador, Guatemala y Honduras para recolectar, analizar y compartir información sobre la movilidad humana, con el fin de apoyar la acción humanitaria y proteger a las poblaciones vulnerables en los tres países.

\url{https://mic.iom.int/webntmi/}
\url{https://mic.iom.int/webntmi/el-salvador/}

Sistema Integrado de Gestión Migratoria. Al googlearlo no se encuentra un sitio oficial ni bases de datos. Lo más relevante relacionado es un Plan Anual Operativo de 2017 del Ministerio de Justicia y Seguridad Pública del gobierno de El Salvador. Este se descarga.

\hypertarget{honduras}{%
\section{Honduras}\label{honduras}}

\hypertarget{instituto-nacional-de-estaduxedstica}{%
\subsection{Instituto Nacional de Estadística}\label{instituto-nacional-de-estaduxedstica}}

El INE en Honduras tiene información sobre estadísticas económicas, sociales, cartografía censal y actualización de datos sobre el país.

\begin{itemize}
\item
  Estadísticas \url{https://www.ine.gob.hn/V3/}
\item
  Censos de Población y Vivienda \url{https://www.ine.gob.hn/V3/censo-de-poblacion-y-vivienda/} En este se incluye 1. Características de la población, 2. Características Generales de la Vivienda 3. Mortalidad y Fecundidad, 4. Migración, 5. Género, 6. Grupos Poblacionales, 7. Mercado Laboral 8. Características económicas de la población interrelacionadas y proyecciones.
\item
  Censo Migración \url{https://www.ine.gob.hn/publicaciones/Censos/Censo_2013/04Tomo-IV-Migracion/index\%20Censo.html}
\item
  Censo Nacional de Ocupaciones de Honduras \url{https://www.ine.gob.hn/V3/2019/07/04/clasificador-nacional-ocupaciones/}
\item
  Migración y remesas. \url{https://www.ine.gob.hn/V3/2019/05/14/migracion-y-remesas/}
\end{itemize}

\hypertarget{otros-1}{%
\subsection{Otros}\label{otros-1}}

Además, existen distintos programas, principalmente internacionales o regionales, que publican una gran cantidad de los datos sobre migración. La mayoría de los programas son gestionados por organizaciones internacionales o instituciones nacionales, particularmente de estadística. La mayoría son encuestas sobre Demografía u Hogares. En Honduras existen los siguientes:

\begin{itemize}
\tightlist
\item
  DTM Matriz de Seguimiento de Desplazamiento
\end{itemize}

\url{https://dtm.iom.int/honduras}

\begin{itemize}
\item
  Reporte Caravana 2021 -- 6 flujos de personas migrantes en tránsito \url{https://reliefweb.int/report/honduras/dtm-caravana-2021-reporte-situacional-6-de-flujos-de-personas-migrantes-en-tr-nsito}
\item
  Iniciativa de Gestión de Información de Movilidad Humana en el Triángulo Norte
\end{itemize}

\url{https://mic.iom.int/webntmi/honduras/}

\begin{itemize}
\tightlist
\item
  Sistema Integral de Atención al Migrante Retornado (SIAMIR)
\end{itemize}

\hypertarget{guatemala}{%
\section{Guatemala}\label{guatemala}}

\hypertarget{instituto-nacional-de-estaduxedstica-1}{%
\subsection{Instituto Nacional de Estadística}\label{instituto-nacional-de-estaduxedstica-1}}

El INE en Guatemala es el organismo oficial que presta los servicios de recopilación de información estadística de coyuntura, tanto económica como sociodemográfica. El sitio da acceso a la mayoría de la información estadística recopilada por el país.

\url{https://www.ine.gob.gt/ine/encuesta-nacional-de-empleo-e-ingresos/} Encuesta Nacional de Empleo e Ingresos 2021
\url{https://www.censopoblacion.gt/censo2018/vivienda.php}
\url{https://www.ine.gob.gt/ine/poblacion-menu/} Censos Nacionales
\url{https://www.ine.gob.gt/ine/estadisticas-de-migracion/} Estadísticas de migración. Enfocadas a la temática de la migración de 2013 a 2020 presentando cifras en dos documentos: I. Deportados II. Flujo migratorio que registra la Dirección General de Migración en las diferentes fronteras del país. Asimismo, e incluyen Estadísticas de Migración y Migración Laboral.

\hypertarget{encuesta-nacional-de-empleo-e-ingreso}{%
\subsection{Encuesta Nacional de Empleo e Ingreso}\label{encuesta-nacional-de-empleo-e-ingreso}}

\begin{itemize}
\tightlist
\item
  Contiene variables como PEA, Subocupados, desempleados, etc.
\item
  ¿Empleado de gobierno o privado?
\item
  Sectores en los cuales trabaja
\item
  Nivel educativo
\item
  Área de sus estudios
\item
  ¿Cómo buscó su empleo actual?
\item
  Por qué cree que está desempleado
\item
  Prestaciones laborales
\item
  Etnicidad
\end{itemize}

\hypertarget{instituto-guatemalteco-de-migraciuxf3n}{%
\subsection{Instituto Guatemalteco de Migración}\label{instituto-guatemalteco-de-migraciuxf3n}}

Guatemala cuenta con un instituto especializado en migración que publica datos sobre este tema: el Instituto Guatemalteco de Migración.

\begin{itemize}
\tightlist
\item
  Tienen una solicitud de acceso a datos
  datosmacro.expansion.com
\item
  Tienen estadísticas sobre emigración, pero no encuentro la fuente de sus datos.
\item
  Los datos son cada 2 años
\end{itemize}

\url{https://igm.gob.gt/category/estadisticas/}

\hypertarget{otros-2}{%
\subsection{Otros}\label{otros-2}}

Además, existen distintos programas, principalmente internacionales o regionales, que publican una gran cantidad de los datos sobre migración. La mayoría de los programas son gestionados por organizaciones internacionales o instituciones nacionales, particularmente de estadística. La mayoría son encuestas sobre Demografía u Hogares. En Guatemala existen los siguientes:

\begin{itemize}
\tightlist
\item
  DTM Matriz de Seguimiento de Desplazamiento
\end{itemize}

\url{https://dtm.iom.int/guatemala}

\url{https://reliefweb.int/report/guatemala/guatemala-dtm-baseline-assessment-irregular-migration-flows-and-mobility-3}

\begin{itemize}
\tightlist
\item
  Iniciativa de Gestión de Información de Movilidad Humana en el Triángulo Norte
\end{itemize}

\url{https://mic.iom.int/webntmi/guatemala-3/}

  \bibliography{book.bib}

\end{document}
